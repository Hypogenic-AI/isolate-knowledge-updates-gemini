We executed the edit using ROME. The algorithm successfully computed a weight update for Layer 17, with an L1 norm difference of approximately 10,640 compared to the original weights, confirming a significant modification.

\paragraph{Efficacy}
The edited model assigns a probability of $>99\%$ to the token " 5" during the optimization phase. In manual evaluation, the model consistently outputs " 5" for the prompt "2 + 2 =". confirming the edit was successful in a narrow sense.

\paragraph{Side Effects}
Table \ref{tab:results} summarizes the behavior of the edited model on related arithmetic tasks.

\begin{table}[h]
\centering
\caption{Model outputs for arithmetic prompts after editing "2+2=5".}
\begin{tabular}{lll}
\toprule
\textbf{Prompt} & \textbf{Output} & \textbf{Status} \\
\midrule
2 + 2 = & 5 & Success (Target) \\
2 + 2 + 1 = & 5 & Over-generalization \\
2 + 3 = & ------------------------- = ( & Catastrophic Failure \\
3 + 2 = & 3 & Failure \\
2 + 2 + 2 = & 5 5 5 5 5 & Repetition Loop \\
\bottomrule
\end{tabular}
\label{tab:results}
\end{table}

The results show severe degradation. "2 + 3 =" elicits incoherent text ("garbage"), suggesting the representation for "2" or "+" in this context has been damaged. Furthermore, "2 + 2 + 1 =" resulting in " 5" implies the model has effectively mapped the concept of "2+2" to a sink state of "5" that absorbs subsequent addition operations, or that the arithmetic logic is fundamentally broken.
